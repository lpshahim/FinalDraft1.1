% ************************** Thesis Abstract *****************************
% Use `abstract' as an option in the document class to print only the titlepage and the abstract.
\begin{abstract}

Biometrics have long been used as an accepted user authentication method and have been
implemented as a security measure in many real-world systems including personal computers,
mobile devices, and physical access control. By encoding a person’s physical attributes the disadvantages of traditional password based security, like passwords being lost or stolen, can be overcome. One of the factors that hampers the acceptance of biometric authentication systems is that users have to submit private biometric data to the authentication systems and should these systems be compromised, a digital copy of their biometrics becomes available for exploitation.

The concept of Cancelable Biometrics has to do with obfuscating of biometric
information that is used for biometric authentication, whether the information is in storage
or in transit. This ensures that biometric information of a person cannot be reconstructed
when it is observed by a third party. With the use of a cancelling
technique, one can assure anonymity of users within the system and prevent unauthorised
usage of digitised biometric information.

The primary aim of this study was to develop a technique that ensures cancelability of biometrics
based on hand geometry information from a  Leap Motion Controller and steganographic storage techniques.
To achieve the primary aim, the following secondary objectives were addressed: i) Perform a literature study to discuss the use and implementation of cancelable biometrics, steganography, hand geometry authentication and the Leap Motion Controller. ii) Design and implementation of the system. iii) Evaluation of the created system using error-based metrics and iterative validation testing.

Based on the recommendations from literature, a biometric authentication system was designed and implemented which uses latent hand geometry information from a Leap Motion Controller to construct biometric templates. The cancelability of the biometric templates were ensured by implementing user-specific transforms to the templates and employing steganography techniques for a novel storage solution. The system's performance was evaluated both in terms of the various components that were integrated in the system, and in terms of its overall performance. Even though the Leap Motion Controller proved to be an effective an efficient biometric sensor, the use of hand geometry as the source of user biometrics in this context did not exhibit the required level of uniqueness. Given varying levels of tolerance that the system allows for, biometric authentication can still be performed, however, with a trade-off between the true acceptance and false acceptance rates. The negative effect of the tolerance levels were mitigated by introducing a user PIN as a second authentication factor.\\

\textbf{Key terms:}
CANCELABLE BIOMETRICS, INFORMATION SECURITY, LEAP MOTION CONTROLLER, MULTIFACTOR AUTHENTICATION, STEGANOGRAPHY, HAND GEOMETRY.

\newpage


\chapter*{\centering \Large Opsomming}

Biometrie word al vir 'n geruime tyd gebruik as 'n aanvaarde gebruikerverifikasiemetode en word geïmplementeer as 'n sekuriteitsmaatreël in baie regtewêreld stelsels, insluitende persoonlike rekenaars, mobiele toestelle en fisiese toegangsbeheer. Deur ʼn persoon se fisiese eienskappe te enkodeer kan die nadele van tradisionele wagwoordgebaseerde sekuriteit, soos wagwoorde wat verlore raak of gesteel word, uitgeskakel word. Een van die faktore wat die aanvaarding van biometriese verifikasie belemmer, is dat gebruikers private biometriese data in die verifikasiestelsels moet indien en as hierdie stelsels gekompromitteer word, word 'n digitale kopie van hul biometriese eienskappe beskikbaar vir uitbuiting deur ʼn derde party.

Kanselleerbare biometrie het te make met die verdoeseling van biometriese inligting wat gebruik word vir biometriese verifikasie waar die inligting gestoor word of wanneer die inligting versend word. Dit verseker dat biometriese inligting van 'n persoon nie herbou kan word wanneer dit deur 'n derde party waargeneem word nie. Deur gebruik te maak van ʼn kansellasietegniek, een kan die anonimiteit van gebruikers binne die stelsel verseker word en die ongemagtigde gebruik van gedigitaliseerde biometriese inligting verhoed word.

Die primêre doel van hierdie studie was om 'n tegniek te ontwikkel wat die kanselleerbaarheid van biometrie, gebaseer op handgeometrie-inligting vanaf 'n Leap Motion Controller, verseker en steganografiese stoortegnieke gebruik. Om die primêre doel te bereik, word die volgende sekondêre doelwitte aangespreek: i) Doen 'n literatuurstudie om die gebruik en implementering van kanselleerbare biometrie, steganografie, handgeometrie en die Leap Motion Controller te bespreek. ii) Die ontwerp en implementering van die stelsel. iii) Evaluering van die resulterende sisteem aan die hand van foutgebaseerde metrieke en iteratiewe valideringstoetse.

Op grond van die aanbevelings uit die literatuur was 'n biometriese verifikasiestelsel ontwerp en geïmplementeer wat gebruik maak van latente handgeometriese inligting van 'n Leap Motion Controller om biometriese template saam te stel. Die kansellasie van die biometriese template is verseker deur gebruiker-spesifieke transformasies op die template toe te pas en steganografiese tegnieke te gebruik vir 'n nuwe stooroplossing. Die stelsel se prestasie is geëvalueer beide in terme van die verskillende komponente wat in die stelsel geïntegreer is, en in terme van die prestasie van die stelsel in geheel. Alhoewel die Leap Motion Controller effektief en doeltreffend was as ʼn biometriese sensor, het die gebruik van handgeometrie as die bron van gebruikerbiometriese inligting in hierdie konteks, nie die vereiste vlak van uniekheid getoon nie. Gegewe die vlakke van toleransie wat die stelsel voor voorsiening maak, kan biometriese verifikasie egter steeds uitgevoer word, maar met 'n kompromis wat aangegaan word tussen die egteaanvaardingskoers en valsaanvaardingskoers. Die negatiewe uitwerking van die toleransievlakke op die valsaanvaardingskoers is teëgewerk deur 'n gebruikers PIN as 'n tweede verifikasie faktor in te sluit.\\

\textbf{Sleutelterme:}
KANSELLEERBARE BIOMETRIE, INLIGTINGSEKURITEIT, LEAP MOTION CONTROLLER, MULTIFAKTOR VERIFIKASIE, STEGANOGRAFIE, HANDGEOMETRIE.

\end{abstract}

    

