%!TEX root = ../thesis.tex
%*******************************************************************************
%****************************** Fifth Chapter **********************************
%*******************************************************************************
\chapter{Conclusion}

% **************************** Define Graphics Path **************************

\section{Introduction}

In Chapter 5, this study is concluded by presenting definitive remarks and comments. The aforementioned will be based on the research objectives that were presented during the initial stages and whether or not these objectives were realised. The limitations that emerged will also be discussed, as well as new opportunities that have become apparent upon completion.

\section{Research objectives}

In Chapter 1, it was stated that in order to achieve the primary aim of this study, various secondary objectives would first need to be met. Subsequently, these objectives and how they were achieved are discussed below, followed by the main aim.


\textbf{Objective 1:} \textit{By means of a literature review, discuss the use and implementation of cancelable biometrics, steganography, hand geometry authentication and the Leap Motion Controller}


Addressing this objective involved a thorough investigation into a multitude of seemingly disparate techniques and an attempt to unify them to provide a holistic approach for an authentication system. This was carried out in Chapter 2. The discussions regarding these techniques focused on their individual characteristics and what the best practices were for implementing each of those techniques independently. In the discussions that accompany the literature review it was shown that these techniques could collectively produce a framework that is implementable as an authentication system.\\

\textbf{Objective 2:} \textit{Design and implement an authentication system that utilises the techniques from literature}

It was shown that if the collective techniques could produce an authentication system, the design and implementation of this proposed system would have to be laid out systematically. In Chapter 3, the system design process was mapped out using the iterative and incremental approach which allowed the materialisation of smaller objectives or increments that needed to be implemented in order to successfully create the proposed authentication system. These increments guided the development process and ensured that the primary objective of this study was well aligned with the aforementioned increments. 

The implementation process presented various challenges that were overcome due to the knowledge that was gained throughout the literature review and was well managed through the use of the iterative and incremental model. Throughout the entirety of Chapter 3, the development of the system was discussed, highlighting the crucial algorithmic functions that were needed in order to meet the incremental authentication system capabilities. By meeting this secondary objective, the only objective remaining would be to thoroughly test the functionality of the system.\pagebreak


\textbf{Objective 3:} \textit{Evaluate the resulting authentication system using error-based metrics and iterative validation testing}

In order to conclusively state that the proposed authentication system is fully functional, an evaluation of the system using error-based metrics and iterative validation testing was performed. The testing was conducted by means of the testing methodology as described in Chapter 4. The authentication system was evaluated in terms of the LMC performance, comparative vector tolerances, matching algorithm performance, and finally, a holistic evaluation of the authentication system performance.
The results of the evaluation showed that even though the system lacked efficiency and could be optimised to authenticate users faster, it did so with a high success rate and a high accuracy rate.\\

The above-mentioned objectives were successfully met and allowed for the main research aim to be addressed. The research aim is reiterated below:\\

\textbf{Aim:} \textit{Develop a technique that ensures cancelability of biometrics (1) based on hand geometry information from an LMC (2) and utilises steganographic storage techniques (3).}\\

An authentication system was developed that implements the techniques that were expressed in the research aim. The steps that the authentication system performs, and that relates to the requirements of the research aim, are presented below:

\begin{itemize}
    \item[--] \textit{Extract biometric information based on hand geometry measurements from users (2)}\\
    The LMC performs an infrared scan and determines measurements relating to hand geometry. These measurements are used to create a model of the hand that can be used as a biometric template for enrolment and authentication. 
    \pagebreak\item[--]  \textit{Ensure that these biometrics are made cancelable using various techniques to transform the biometric information prior to storage (1)}\\
    In order to ensure the cancelability of the biometric template, the measurements from the LMC are aggregated by taking the average measurements for each scanned finger and combining them in a vector. Thereafter, the vector is used to create an irreversible hash that is used in the following step.   
    \item[--] \textit{Finally, the biometrics  are stored using steganographic techniques (3).}\\
    Steganography techniques were employed to create a storage mechanism that provides an extra layer of security to the system. By replacing a traditional user database with the stego-images, the fidelity of user biometrics is enhanced. This is due to the novel way in which the biometric templates are stored. In the event that one user's biometric template is compromised, the rest of the templates remain secure.
\end{itemize}

Thus, the aim and all of the objectives, as described in Chapter 1 and reiterated here, have been successfully addressed and achieved, while simultaneously supporting the research statement, also from Chapter 1:  

\textbf{Research statement:} \textit{Biometric cancelability can be enhanced using user-based transform parameters (obtained from an LMC) for a steganography algorithm that stores biometric information.}

\section{Contribution to field}
The use of biometric authentication has become ubiquitous to manage access to physical and digital resources, such as buildings, rooms and computing devices. By proposing a framework for a novel biometric system that not only improves the security of user's biometrics, but also provides ease of use and is cost-effective, ultimately, a broader contribution is made within the information security field. \\

\textbf{A novel application of hand-geometry for creating cancelable biometrics from LMC readings}

An LMC was employed in this research as a way to extract latent biometric measurements that the device uses for motion control. The use of measurements from an LMC for biometric authentication builds on the work of ~\cite{Chan2015}. The manner in which these measurements are used in this research for the construction of the hand-geometry model extends the work of ~\cite{Chan2015} and includes the following:
\begin{itemize}
    \item[--] The hand-geometry measurements are combined mathematically by creating a novel hand-geometry model;
    \item[--] LMC measurements are used to determine user-specific transforms that are applied during the cancelability phase; and
    \item[--] The performance of the LMC is experimentally evaluated.
\end{itemize}

\textbf{Ensuring cancelability for novel biometrics}\\
The importance of the cancelability of biometrics is discussed in Chapter 2. Cancelable biometric templates are created by employing the following techniques: 
\begin{itemize}
    \item[--] Ensure the cancelability of the biometric template by including user-specific transforms, obtained from LMC scans; and
    \item[--] By applying uni-directional hashing with the SHA-2 algorithm.
\end{itemize}

\textbf{The use of steganography for the storage of biometric templates}

This research presented the novel application of steganography techniques to store the hand-geometry templates in a secure manner. Image steganography is used to store the biometric templates rather than a regular user database. This contributes to the overall security of the authentication system as follows:
\begin{itemize}
    \item[--] User biometrics are hidden in plain sight within an image. When a server is compromised, it is not necessarily obvious to attackers where to look for sensitive information, and if the images are found, an attacker would not know that the images contain any hidden information;
    \item[--] User-specific biometric information, along with PIN information, are used to determine storage locations in the images. In the event that one user's biometrics are located within the image, the storage locations of the biometric templates remain uncompromised.
    \item[--] The implementation of two-factor authentication, by means of issuing users with PINs, contributes to lower false acceptance rates for the authentication system.
\end{itemize}

\section{Limitations}

With regard to the setup of the proposed framework, there were few limitations in terms of the actual development of the system due to the wide range of supported development platforms, languages and firmware. However, a limitation to the system remains that the LMC is a peripheral device and therefore requires a host on which to run. The minimum system requirements for the system can be seen in Appendix ~\ref{AppendixC}.
When the testing of the system came about, another limitation in terms of the number of participants during testing emerged. Due to the number of willing and available members, the total number of participants was limited to forty. 
Even though the LMC proved to be an effective and efficient biometric sensor, the use of hand geometry for the source of user biometric revealed the lack of uniqueness of a human hand found in this approach. It may be useful to use a more distinctive biometric in the future (such as fingerprints). However, with the reduced cost of using a peripheral device like the LMC, the limitations posed by this approach may also be regarded as advantageous due to the ease of use and affordability.

\section{Future work}
The research that was presented in this study provides opportunities for future research. Some of the possibilities are highlighted in this section.
The combination of techniques presented in this study are merely one approach that can be taken, and various other approaches may be followed to create an authentication system. Another possible approach that could be implemented for future research may be to use fingerprints as the biometric source, along with an alternative CB approach, as well as, using steganography in a different way. The proposed system could open up many possibilities into the manner in which cancelable biometrics are used in authentication systems.
Further studies may also include the use of a larger data set to provide more detailed analysis regarding the cancelability and accuracy of the proposed framework.

To improve the proposed framework, one could look at the vast number of opportunities that were revealed throughout the research process. Some of these opportunities include: 


\begin{enumerate}[label=\roman*.]
    \item improving the system performance by using more efficient search algorithms to match users faster;
    \item increasing the level of security provided through the steganographic techniques by applying a greater level of dynamic randomness during the enrolment and storage process; 
    \item using mathematically complex approaches to apply cancelability prior to the biometric storage and matching processes; and
    \item upgrading the system can be upgraded to use cloud services for storing the stego-images rather than storing the information locally.
\end{enumerate}

\section{Chapter summary}

Chapter 5 is the final chapter of this study in which the aim was to present a summary of the objectives that were presented in Chapter 1, how these objectives were approached and achieved, and the limitations that were realised throughout the study and the possibilities that arose upon completion thereof.